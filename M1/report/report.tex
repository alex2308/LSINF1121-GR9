\documentclass{scrartcl}
\usepackage[utf8]{inputenc}
\usepackage[T1]{fontenc}      
\usepackage[francais]{babel}
\usepackage{algorithm2e}
% Layout and figures
\usepackage[top=2.5cm,bottom=2.5cm,right=2.5cm,left=2.5cm]{geometry}
% Links
\usepackage{url}
\usepackage{hyperref}
\hypersetup{
    colorlinks,
    citecolor=black,
    filecolor=black,
    linkcolor=black,
    urlcolor=black
}

% New commands
\newcommand{\annexe}{\part{Annexes}\appendix}
\newcommand{\biblio}[1]{\bibliographystyle{plain}\bibliography{#1}\nocite{*}}

\newcommand{\doctitle}[1]{
	\title{LINGI1121 - Algorithmique et structure de données}
	\subtitle{#1}
	\author{\textbf{Groupe 9}\\
	\textsc{Aghakhani} Ghazaleh (1161-11-00)\\
	\textsc{Carlier} Alexandre (5042-13-00)\\
	\textsc{Cleeremans} Tanguy ()\\
	\textsc{Paris} Antoine (3158-13-00)\\
	\textsc{Prie\"{e}ls} Antoine (3290-13-00)}\\
	\textsc{Stévenart Meeus} Florian (6273-13-00)}
	\date{\today}

	\begin{document}

	\maketitle
	%\tableofcontents
}
\doctitle{Mission 1 : piles, files, listes chaînées}
\lstset{language={Java}}
\lstset{
  numbers=left,
  numberstyle=\tiny\color{gray},
  basicstyle=\rm\small\ttfamily,
  keywordstyle=\bfseries\color{black},
  frame=single,
  commentstyle=\color{gray}=small,
  stringstyle=\color{dkgreen},
  %backgroundcolor=\color{gray!10},
  %tabsize=2,
  rulecolor=\color{black},
  %title=\lstname,
  breaklines=true,
  framextopmargin=2pt,
  framexbottommargin=2pt,
  extendedchars=true,
}

\section{Réponses aux questions}
\begin{enumerate}
	\item Un type abstrait est une spécification
	d'un ensemble de données et de l'ensemble des
	opérations qu'on peut lui appliquer (sans se
	préoccuper de l'implémentation de celles-ci).
	Il s'agit en quelque sorte d'un cahier des
	charges pour une structure de données.
	\cite{wiki-tad}\cite{mod-obj}
	
	Quant au choix d'utiliser une classe ou une
	interface en Java pour décrire un type abstrait
	de données, il semble tout indiqué. Une interface
	Java n'a pour but que de définir une implémentation ;
	elle est par définition abstraite et ne fixe aucun
	aspect de l'implémentation.\cite{nino} Cela colle
	parfaitement à la définition de type abstrait de
	données.
	\item L'opération \lstinline{push}\lstinline{} pour une pile
	implémentée à l'aide d'une liste s'implémente en
	parcourant toute la liste jusqu'à arriver à la fin
	de la liste. \`{A} ce moment là, on modifie le
	pointeur vers l'élément suivant du dernier élement
	pour le faire pointer vers le nouvel élément.
	
	L'opération \lstinline{pop}\lstinline{} s'implément à peu près
	de la même façon. On parcourt la liste jusqu'à
	arriver à la fin. \`{A} ce moment là, on retourne
	le dernier élément de la liste et on modifie le
	pointeur de l'avant-dernier élément pour le faire
	pointer vers \lstinline{null}\lstinline{}.
	
	Cette implémentation n'est pas efficace parce qu'il
	faut parcourir toute la liste à chaque opération, ce
	qui peut vite devenir gênant pour des listes contenant
	des millions d'éléments...
	\item L'implémentation d'une pile par la classe
	\lstinline{java.util.Stack}\lstinline{} fournit 5
	fonctions :
	\begin{itemize}
		\item \lstinline{boolean empty()}\lstinline{} :
		teste si la pile est vide ;	
		\item \lstinline{E peek()}\lstinline{} : retourne
		l'objet au sommet de la pile sans l'enlever ;
		\item \lstinline{E pop()}\lstinline{} : retourne
		l'objet au sommet de la pile en l'enlevant ;
		\item \lstinline{E push(E item)}\lstinline{} :
		ajoute un élément au sommet de la pile ;
		\item \lstinline{int search(Object o)}\lstinline{} :
		retourne la position de l'objet dans la pile.
	\end{itemize}
	
	En regardant le code source de cette classe, on
	constate que la plupart des fonctions sont héritées
	de la classe \lstinline{Vector<E>}\lstinline{}. En
	allant ensuite regarder le code source de cette classe,
	on se rend compte que celle-ci utilise un tableau pour
	stocker des éléments ainsi qu'un compteur d'éléments. 
	Cela signifie donc qu'en Java, les éléments d'une liste
	chaînée sont stockés dans un tableau. Cette solution
	est beaucoup plus efficace que celle proprosé à la
	question précédente puisqu'il est possible d'accèder au
	dernier élément de la liste sans la parcourir entièrement.
	\item La méthode la plus efficace pour implémenter
	une pile avec deux files provient de \cite{stack1}.
	Soient deux piles $A$ et $B$. $A$ contient les
	éléments au sommet de la pile tandis que $B$
	contient les éléments du bas de la pile. La taille
	de $A$ doit toujours être inférieur à la
	racine carrée de la taille de $B$.
	
	\lstinline{push}\lstinline{} s'effectue simplement
	en effectuant \lstinline{enqueue}\lstinline{} 
	du nouvel élément sur la pile $A$ et ensuite en
	effectuant \lstinline{dequeue}\lstinline{} puis
	\lstinline{enqueue}\lstinline{} sur tous les autres
	élément de $A$. De cette manière le nouvel élément
	est bien le premier élément de $A$.
	
	Si le nombre d'élément contenu dans $A$ devient
	plus grand que la racine carrée du nombre
	d'élément contenu dans $B$, on
	\lstinline{enqueue}\lstinline{} tous les éléments
	de $B$ sur $A$ un par un et on inverse $A$ et $B$.
	
	Enfin, \lstinline{pop}\lstinline{} s'effectue en
	effectuant \lstinline{dequeue}\lstinline{} sur $A$
	et en retournant le résultat si $A$ n'est pas vide
	et en effectuant \lstinline{dequeue}\lstinline{}
	sur $B$ dans le cas contraire.
	
	En terme de complexité,
	\lstinline{pop}\lstinline{} s'effectue en
	$\mathcal{O}(1)$.
	
	Pour \lstinline{push}\lstinline{}, deux cas sont
	à analyser. Dans le premier cas,
	$|A| < \sqrt{|B|}$ et on a donc
	$\mathcal{O}(\sqrt{n})$. Dans le cas contraire,
	\lstinline{push}\lstinline{} s'effectue en
	$\mathcal{O}(n)$ mais après cela, $A$ est vide et
	il faudra un temps $\mathcal{O}(\sqrt{n})$ avant
	que ce cas ne se reproduise, le coût amorti est
	donc en $\mathcal{O}(\sqrt{n})$.
	\item
	\item
	\item
	\item
	\item
	\item
	\item
\end{enumerate}

\bibliographystyle{plain}
\bibliography{./biblio}

\end{document}