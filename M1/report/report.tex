\documentclass{scrartcl}
\usepackage[utf8]{inputenc}
\usepackage[T1]{fontenc}      
\usepackage[francais]{babel}
\usepackage{algorithm2e}
% Layout and figures
\usepackage[top=2.5cm,bottom=2.5cm,right=2.5cm,left=2.5cm]{geometry}
% Links
\usepackage{url}
\usepackage{hyperref}
\hypersetup{
    colorlinks,
    citecolor=black,
    filecolor=black,
    linkcolor=black,
    urlcolor=black
}

% New commands
\newcommand{\annexe}{\part{Annexes}\appendix}
\newcommand{\biblio}[1]{\bibliographystyle{plain}\bibliography{#1}\nocite{*}}

\newcommand{\doctitle}[1]{
	\title{LINGI1121 - Algorithmique et structure de données}
	\subtitle{#1}
	\author{\textbf{Groupe 9}\\
	\textsc{Aghakhani} Ghazaleh (1161-11-00)\\
	\textsc{Carlier} Alexandre (5042-13-00)\\
	\textsc{Cleeremans} Tanguy ()\\
	\textsc{Paris} Antoine (3158-13-00)\\
	\textsc{Prie\"{e}ls} Antoine (3290-13-00)}\\
	\textsc{Stévenart Meeus} Florian (6273-13-00)}
	\date{\today}

	\begin{document}

	\maketitle
	%\tableofcontents
}
\doctitle{Mission 1 : piles, files, listes chaînées}
\lstset{language={Java}}
\lstset{
  numbers=left,
  numberstyle=\tiny\color{gray},
  basicstyle=\rm\small\ttfamily,
  keywordstyle=\bfseries\color{dkred},
  frame=single,
  commentstyle=\color{gray}=small,
  stringstyle=\color{dkgreen},
  %backgroundcolor=\color{gray!10},
  %tabsize=2,
  rulecolor=\color{black},
  %title=\lstname,
  breaklines=true,
  framextopmargin=2pt,
  framexbottommargin=2pt,
  extendedchars=true,
}

\section{Réponses aux questions}
\begin{enumerate}
	\item Un type abstrait est une spécification
	d'un ensemble de données et de l'ensemble des
	opérations qu'on peut lui appliquer (sans se
	préoccuper de l'implémentation de celles-ci).
	Il s'agit en quelque sorte d'un cahier des
	charges pour une structure de données.
	\cite{wiki-tad}\cite{mod-obj}
	
	Quant au choix d'utiliser une classe ou une
	interface en Java pour décrire un type abstrait
	de données, il semble tout indiqué. Une interface
	Java n'a pour but que de définir une implémentation ;
	elle est par définition abstraite et ne fixe aucun
	aspect de l'implémentation.\cite{nino} Cela colle
	parfaitement à la définition de type abstrait de
	données.
	\item L'opération \lstinline{push} et \lstinline{pop}
	pour une pile implémentée à l'aide d'une liste simplement
	chaînée est donnée en pseudo-code :
	
	\begin{algorithm}
		
	\end{algorithm}

	\begin{algorithm}
		
	\end{algorithm}

	\item
	\item
	\item
	\item
	\item
	\item
	\item
	\item
	\item
\end{enumerate}

\bibliographystyle{plain}
\bibliography{./biblio}

\end{document}